\documentclass[10pt]{article}

\usepackage{amssymb, amsfonts, amsmath, amsthm}
\usepackage{mathrsfs}
\usepackage{hyperref}
\usepackage{graphicx}
\usepackage{caption}
\usepackage{subcaption}
\usepackage{wrapfig}
\usepackage[margin=1.00 in]{geometry}
\usepackage{enumerate}
\usepackage{harpoon}
\usepackage{nicefrac}
\usepackage{tikz, pgfplots}
\usepackage{xypic}
\usepackage{tikz-cd}
\usepackage{booktabs}

\usepackage{fancyhdr}
\pagestyle{fancy}
\fancyhead{} % clear all header fields
\renewcommand{\headrulewidth}{0pt} % no line in header area
\fancyfoot{} % clear all footer fields
\fancyfoot[LE,RO]{\footnotesize \thepage}           % page number in "outer" position of footer line

\pgfplotsset{compat=newest}

%\setlength\parindent{0pt}%removes indents from entire file

\newtheorem*{sol}{Solution}

\newcommand{\N}{\mathbb{N}}
\newcommand{\Z}{\mathbb{Z}}
\newcommand{\R}{\mathbb{R}}
\newcommand{\C}{\mathbb{C}}
\renewcommand{\P}{\mathbb{P}}

\begin{document}

\noindent \large{Solutions to selected exercises from Chapter 1 of
\emph{Wasserman --- All of Statistics}}

\begin{enumerate}
\item[(5)]

The sample space is
\[\Omega = \{HH, THH, HTH, HTTH, THTH, TTHH, HTTTH, THTTH, \ldots\}.\]
I.e. an element of $\Omega$ is a sequence in $T$ and $H$
of length $k$ with the last element
being $H$ and exactly one of the first $k-1$ elements being $H$.

There are exactly $k-1$ sequences in $\Omega$ of length $k$, corresponding
to the $k-1$ choices for where to put an $H$ in the initial sequence of
length $k-1$. Each length $k$ sequence has probability $(1/2)^k$ of occurring.
Hence the probability that exactly $k$ tosses are required is $(k-1)(1/2)^k$

\item[(8)]

We have $\P(\bigcap A_i) = 1 -\P(\bigcup A_i^c) \geq 1 - \sum \P(A_i^c)$
where $\cdot^c$ denotes the complement. Since $\P(A_i^c)=0$ for each $i$
we have $\P(\bigcap A_i)=0$, as claimed.

\item[(11)]
We have $\P(A^cB^c)=\P((A\cup B)^c)=1-\P(A\cup B)=1-\P(A)-\P(B)+\P(AB)$.
Since $A$ and $B$ are independent this is
\[
1-\P(A)-\P(B)+\P(A)\P(B)=(1-\P(A))(1-\P(B)) = \P(A^c)\P(B^c).
\]


\item[(12)]
The sample space of possible sequences of sides that we can see is
$\Omega = \{RR, GG, RG, GR\}$ where $R$ denotes a red side and $G$ denotes
a green side. The elements of $\Omega$ have probability $1/3$, $1/3$,
$1/6$, and $1/6$. Hence the probability that the other side is green, given
that the first side we see is green is
\[
\frac{1/3}{1/3+1/6} = \frac{2}{3}.
\]

\item[(13)]
\begin{enumerate}[(a)]
\item The sample space is
\[
\Omega = \{HT, TH, HHT, TTH, HHHT, TTTH, HHHHT, TTTTH, \ldots\}
\]
I.e. an element of $\Omega$ is a sequence of length $n$ where the first
$n-1$ elements are $H$ and the last element is $T$, or vice versa.
In particular, for any $n\geq 2$ there are exactly two sequences in $\Omega$
of length $n$.

\item The probability of three tosses being required is $2(1/2)^3=1/4$ since
there are two elements of $\Omega$ of length $3$, both with probability
$(1/2)^3$.
\end{enumerate}

\item[(15)]

\begin{enumerate}[(a)]
\item Order the children from youngest to oldest. Thus the sample space
consists of sequences of length three in $B$ and $N$, $B$ indicating
that a child has blue eyes and $N$ indicating that they do not have blue
eyes:
\[
\Omega = \{NNN, BNN, NBN, NNB, BBN, BNB, NBB, BBB\}.
\]
If at least one child has blue eyes then the element of the sample space
$\Omega$ can be anything except for $NNN$. The probability of this occuring
is \[
3(1/4)(3/4)^2 + 3(1/4)^2(3/4) + (1/4)^3.
\]
Here the first term is the probability of one of the sequences $BNN,NBN,NNB$,
the second term is the probability of one of the sequences $BBN, BNB, NBB$,
and the third term is the probability of $BBB$. So the probability that
at least two children have blue eyes is
\[3(1/4)^2(3/4) + (1/4)^3.\]
The probability that at least two children have blue eyes given that one child
has blue eyes is
\[
\frac{3(1/4)^2(3/4) + (1/4)^3}{3(1/4)(3/4)^2 + 3(1/4)^2(3/4) + (1/4)^3}
=\frac{10}{37}.
\]
\item If the first child has blue eyes then the sequence is
$BNN, BBN, BNB$, or $BBB$. The probability for this event is
$(1/4)(3/4)^2 + 2(1/4)^2(3/4) + (1/4)^3$. The probability that the first child
has blue eyes and at least two children have blue eyes is
$2(1/4)^2(3/4) + (1/4)^3$. So the conditional probability is
\[
\frac{2(1/4)^2(3/4) + (1/4)^3}{(1/4)(3/4)^2 + 2(1/4)^2(3/4) + (1/4)^3}
=\frac{7}{16}.
\]
\end{enumerate}

\item[(19)]

Let $M$ stand for Mac, $W$ for Windows, $L$ for Linux, and $V$ for virus.
By Bayes's Theorem,
\[
\P(W|V) = \frac{\P(V|W)\P(W)}{\P(V|M)\P(M) +\P(V|W)\P(W) + \P(V|L)\P(L)}.
\]
Substituting yields that $\P(W|V)$ is
\[
\frac{(82/100)(50/100)}{(65/100)(30/100)+(82/100)(50/100)+(50/100)(20/100)}
= \frac{82}{141} \approx 58.156 \%
\]

\item[(20)]

\begin{enumerate}[(a)]
\item By Bayes's Theorem:
\[
\P(C_i | H) = \frac{\P(H|C_i)\P(C_i)}{\sum_j \P(H | C_j)\P(C_j)}=
\frac{p_i(1/5)}{\sum_j p_j(1/5)}=\frac{p_i}{1/4+1/2+3/4+1} = \frac{2p_i}{5}.
\]
I.e. the probabilities are
\[
0, \ 1/10, \ 1/5, \ 3/10, \ 2/5.
\]
\item We have $\P(H_2|H_1)=\P(H_1H_2)/\P(H_1)$ and $\P(H_1)=\sum p_i/5 =1/2$.
We also have
\[
\P(H_1H_2) = \sum_{i=1}^5 \P(H_1H_2C_i) = \sum_i \P(H_1H_2|C_i)\P(C_i)
=1/5 \sum_i p_i^2 =3/8.
\]
Therefore
\[
\P(H_2|H_1) = \frac{3/8}{1/2} = \frac{3}{4}.
\]
\item By Bayes's Theorem:
\[
\P(C_i|B_4)= \frac{\P(B_4|C_i)\P(C_i)}{\sum_j \P(B_4|C_j)\P(C_j)}
=\frac{(1-p_i)^3p_i(1/5)}{\sum_j(1-p_j)^3p_j(1/5)} =
\frac{128(1-p_i)^3p_i}{23}.
\]
The probabilities are
\[
0, \ 27/46, \ 8/23, \ 3/46, \ 0.
\]
\end{enumerate}

\item[(22)]

See the Jupyter Notebook
\href{https://github.com/ajrasmus/some_of_statistics/blob/main/chapter_1/22.ipynb}{22.ipynb}.

\item[(23)]

See the Jupyter Notebook
\href{https://github.com/ajrasmus/some_of_statistics/blob/main/chapter_1/23.ipynb}{23.ipynb}.

\end{enumerate}
\end{document}
