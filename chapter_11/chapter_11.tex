\documentclass[10pt]{article}

\usepackage{amssymb, amsfonts, amsmath, amsthm}
\usepackage{mathrsfs}
\usepackage{hyperref}
\usepackage{graphicx}
\usepackage{caption}
\usepackage{subcaption}
\usepackage{wrapfig}
\usepackage[margin=1.00 in]{geometry}
\usepackage{enumerate}
\usepackage{harpoon}
\usepackage{nicefrac}
\usepackage{tikz, pgfplots}
\usepackage{xypic}
\usepackage{tikz-cd}
\usepackage{booktabs}
\usepackage{systeme}


\usepackage{fancyhdr}
\pagestyle{fancy}
\fancyhead{} % clear all header fields
\renewcommand{\headrulewidth}{0pt} % no line in header area
\fancyfoot{} % clear all footer fields
\fancyfoot[LE,RO]{\footnotesize \thepage}           % page number in "outer" position of footer line

\pgfplotsset{compat=newest}

%\setlength\parindent{0pt}%removes indents from entire file

\newtheorem*{sol}{Solution}

\newcommand{\N}{\mathbb{N}}
\newcommand{\Z}{\mathbb{Z}}
\newcommand{\R}{\mathbb{R}}
\newcommand{\C}{\mathbb{C}}
\renewcommand{\P}{\mathbb{P}}
\newcommand{\Unif}{\operatorname{Uniform}}
\newcommand{\Bern}{\operatorname{Bernoulli}}
\newcommand{\Binom}{\operatorname{Binomial}}
\newcommand{\Poiss}{\operatorname{Poisson}}
\newcommand{\se}{\operatorname{se}}
\newcommand{\V}{\mathbb{V}}
\newcommand{\E}{\mathbb{E}}
\newcommand{\Cov}{\operatorname{Cov}}
\newcommand{\ol}{\overline}

\begin{document}

\noindent \large{Solutions to selected exercises from Chapter 11 of
\emph{Wasserman --- All of Statistics}}

\begin{enumerate}[(1)]
\item[(1)]

\item[(2)]
See the Jupyter Notebook
\href{https://github.com/ajrasmus/some_of_statistics/blob/main/chapter_11/2.ipynb}{2.ipynb}.


\item[(3)] Set $M=\max\{x_1,\ldots,x_n\}$. Then
\[
    \mathcal L(\theta) = \begin{cases}
        0 & \text{if } \theta < M \\
        \theta^{-n} & \text{if } \theta \geq M \\
    \end{cases}.
\]
Thus,
\[
    f(\theta | x^n) \propto f(\theta)\mathcal L(\theta) =
    \begin{cases}
        0 & \text{if } \theta < M \\
        \theta^{-n-1} & \text{if } \theta \geq M
    \end{cases}.
\]
The integral is $\int_M^\infty \theta^{-n-1} = \frac{1}{n}M^{-n}$.
Hence
\[
    f(\theta|x^n) = \begin{cases}
        \frac{n\max\{x_1,\ldots,x_n\}^n}{\theta^{n+1}} &
        \text{if } \theta \geq \max\{x_1,\ldots,x_n\} \\
        0 & \text{else }
    \end{cases}
\]

\item[(4)]
\end{enumerate}
\end{document}
