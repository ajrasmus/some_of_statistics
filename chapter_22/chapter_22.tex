\documentclass[10pt]{article}

\usepackage{amssymb, amsfonts, amsmath, amsthm}
\usepackage{mathrsfs}
\usepackage{hyperref}
\usepackage{graphicx}
\usepackage{caption}
\usepackage{subcaption}
\usepackage{wrapfig}
\usepackage[margin=1.00 in]{geometry}
\usepackage{enumerate}
\usepackage{harpoon}
\usepackage{nicefrac}
\usepackage{tikz, pgfplots}
\usepackage{xypic}
\usepackage{tikz-cd}
\usepackage{booktabs}

\usepackage{fancyhdr}
\pagestyle{fancy}
\fancyhead{} % clear all header fields
\renewcommand{\headrulewidth}{0pt} % no line in header area
\fancyfoot{} % clear all footer fields
\fancyfoot[LE,RO]{\footnotesize \thepage}           % page number in "outer" position of footer line

\pgfplotsset{compat=newest}

%\setlength\parindent{0pt}%removes indents from entire file

\newtheorem*{sol}{Solution}

\newcommand{\N}{\mathbb{N}}
\newcommand{\Z}{\mathbb{Z}}
\newcommand{\R}{\mathbb{R}}
\newcommand{\C}{\mathbb{C}}
\renewcommand{\P}{\mathbb{P}}
\newcommand{\Unif}{\operatorname{Uniform}}
\newcommand{\Bern}{\operatorname{Bernoulli}}
\newcommand{\Binom}{\operatorname{Binomial}}
\newcommand{\Poiss}{\operatorname{Poisson}}
\newcommand{\se}{\operatorname{se}}
\newcommand{\V}{\mathbb{V}}
\newcommand{\E}{\mathbb{E}}
\newcommand{\Cov}{\operatorname{Cov}}
\newcommand{\ol}{\overline}

\begin{document}

\noindent \large{Solutions to selected exercises from Chapter 22 of
\emph{Wasserman --- All of Statistics}}

\begin{enumerate}

\item[(3)]
See the Jupyter Notebook
\href{https://github.com/ajrasmus/some_of_statistics/blob/main/chapter_22/3.ipynb}{3.ipynb}.


\item[(4)]
The VC-dimension of $\mathcal A$ is 3. One may check that any set of
three non-collinear points is shattered by $\mathcal A$.

So we consider
four points in $\R^2$ and we claim they cannot be shattered. If three
of the points, say $x,y,z$ are collinear, then one of them, say $y$,
lies on the line segment $[x,z]$. On the other hand, if $A\in \mathcal A$
is a disk containing $x,z$ then it also contains $[x,z]$ by convexity
and hence it contains $y$. Otherwise we have four points $x,y,z,w$ in
general position. The set of line segments between points in $\{x,y,z,w\}$
is a quadrilateral with line segments joining its two pairs of opposite
vertices. Thus two such line segments cross. Say $[x,y]$ and $[z,w]$
cross. Suppose there is a disk $A$ containing $\{x,y\}$ but not $\{z,w\}$
and a disk $B$ containing $\{z,w\}$ but not $\{x,y\}$. The intersection of
$A$ and $B$ is a lens $L$ bounded by an arc of the boundary $\partial A$
and an arc of the boundary $\partial B$. Denote the two points of intersection
of these arcs by $p$ and $q$. Denote the bi-infinite line through $p$ and $q$
by $\mathcal L$. Then one side of $\mathcal L$ contains no point of
$A\setminus B$ and the other side contains no point of $B\setminus A$.
Hence $[x,y]$ lies on one side of $\mathcal L$ and $[z,w]$ lies on the other
side so that $[x,y]$ and $[z,w]$ cannot cross. This is a contradiction.

\item[(5)] See the Jupyter Notebook
\href{https://github.com/ajrasmus/some_of_statistics/blob/main/chapter_22/5.ipynb}{5.ipynb}.

\item[(6)] See the Jupyter Notebook
\href{https://github.com/ajrasmus/some_of_statistics/blob/main/chapter_22/6.ipynb}{6.ipynb}.

\item[(7)]
It is clear that no linear classifier can perfectly classify the data assuming
there are some $i$ falling into the three different cases
$X_i<-1$, $-1\leq X_i\leq 1$, and $X_i>1$. On the other hand,
the data $Z_i$ can be separated by the plane $y=1$ in $\R^2$.

\item[(8)] See the Jupyter Notebook
\href{https://github.com/ajrasmus/some_of_statistics/blob/main/chapter_22/8.ipynb}{8.ipynb}.

\item[(9)] Apply the $k$ nearest neighbors classifier to the ``iris data.''
Choose $k$ by cross-validation.

\item[(10)]
This is actually the formula for the median distance for $n$ points in the
\emph{unit ball}. See e.g. Hastie-Tibshirani-Friedman equation (2.24). Moreover,
the correct expression for the median is actually $(1 - (1/2)^{1/n})^{1/d}$ (i.e.
there is no need for the factor $v_d(1)^{-1/d}$).

To prove this equation for the unit ball, note that if $X_1,\ldots,X_n$
are uniformly distributed, then for any $r \in [0,1]$
\[
    \P(R>r) = \P(|X_i| > r \text{ for all } i).
\]
The volume of the $r$-ball is $v_d(r)=r^d v_d(1)$, and renormalizing to give
the unit ball volume 1 by dividing by $v_d(1)$, the volume is just $r^d$.
Therefore
\[
    \P(R>r) = 1 - (1-r^d)^n \text{ and } \P(R \leq r) = (1-r^d)^n.
\]
The median is given by solving
\[
    (1-r^d)^n = \frac{1}{2}
\]
which yields $r=\left(1-\frac{1}{2^{1/n}}\right)^{1/d}$ as claimed.

I'm not sure what the correct expression is for the median closest distance
for the cube $[-1/2,1/2]^d$, but it seems pretty tedious to calculate.

\item[(11)]
See the Jupyter Notebook
\href{https://github.com/ajrasmus/some_of_statistics/blob/main/chapter_22/11.ipynb}{11.ipynb}.

\item[(12)] See the Jupyter Notebook
\href{https://github.com/ajrasmus/some_of_statistics/blob/main/chapter_22/12.ipynb}{12.ipynb}.
\end{enumerate}
\end{document}
