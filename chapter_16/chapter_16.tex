\documentclass[10pt]{article}

\usepackage{amssymb, amsfonts, amsmath, amsthm}
\usepackage{mathrsfs}
\usepackage{hyperref}
\usepackage{graphicx}
\usepackage{caption}
\usepackage{subcaption}
\usepackage{wrapfig}
\usepackage[margin=1.00 in]{geometry}
\usepackage{enumerate}
\usepackage{harpoon}
\usepackage{nicefrac}
\usepackage{tikz, pgfplots}
\usepackage{xypic}
\usepackage{tikz-cd}
\usepackage{booktabs}
\usepackage{systeme}


\usepackage{fancyhdr}
\pagestyle{fancy}
\fancyhead{} % clear all header fields
\renewcommand{\headrulewidth}{0pt} % no line in header area
\fancyfoot{} % clear all footer fields
\fancyfoot[LE,RO]{\footnotesize \thepage}           % page number in "outer" position of footer line

\pgfplotsset{compat=newest}

%\setlength\parindent{0pt}%removes indents from entire file

\newtheorem*{sol}{Solution}

\newcommand{\N}{\mathbb{N}}
\newcommand{\Z}{\mathbb{Z}}
\newcommand{\R}{\mathbb{R}}
\newcommand{\C}{\mathbb{C}}
\renewcommand{\P}{\mathbb{P}}
\newcommand{\Unif}{\operatorname{Uniform}}
\newcommand{\Bern}{\operatorname{Bernoulli}}
\newcommand{\Binom}{\operatorname{Binomial}}
\newcommand{\Beta}{\operatorname{Binomial}}
\newcommand{\Poiss}{\operatorname{Poisson}}
\newcommand{\Gamm}{\operatorname{Gamma}}
\newcommand{\se}{\operatorname{se}}
\newcommand{\V}{\mathbb{V}}
\newcommand{\E}{\mathbb{E}}
\newcommand{\Cov}{\operatorname{Cov}}
\newcommand{\ol}{\overline}

\begin{document}

\noindent \large{Solutions to selected exercises from Chapter 16 of
\emph{Wasserman --- All of Statistics}}

\begin{enumerate}[(1)]
\item[(1)] Suppose the data are given by the following table

\begin{center}
\begin{tabular}{l l l l}
$X$ & $Y$ & $C_0$ & $C_1$ \\
\hline
\hline
0 & 0 & 0 & $0^*$ \\
0 & 0 & 0 & $0^*$ \\
\hline
1 & 0 & $1^*$ & 0 \\
1 & 1 & $1^*$ & 1 \\
\end{tabular}
\end{center}
Then $\theta = \E(C_1) - \E(C_0)=1/4-1/2=-1/4$. On the other hand
$\alpha = \E(Y|X=1)-\E(Y|X=0)=1/2 - 0 = 1/2$.

\item[(2)] Saying that $X$ is randomly assigned means that $X$ is
independent of the collection $\{C(x):x\in \R\}$. Thus in this case
\[
    \theta(x)=\E(C(x)) = \E(C(x)|X=x)=\E(Y|X=x)=r(x).
\]
To see that $\theta \neq r$ in general, suppose that $X\sim N(0,1)$ and that
for each $x\in \R$, $C(x)$ is the random variable $X$. Then $\theta(x)=\E(X)=0$.
However, $r(x)=\E(Y|X=x)=\E(X|X=x)=x$.

\item[(3)] We have
\[
    \E(C_1)=\E(C_1|X=1)\P(X=1) + \E(C_1|X=0)\P(X=0).
\]
Since $\E(C_1|X=1)=\E(Y|X=1)$, we have
\[
    \E(Y|X=1)\P(X=1) \leq \E(C_1)\leq \E(Y|X=1)\P(X=1) + \P(X=0).
\]
Similarly,
\[
    \E(Y|X=0)\P(X=0) \leq \E(C_0) \leq \E(Y|X=0)\P(X=0)+\P(X=1).
\]
Thus,
\[
    \E(C_0)\geq \E(Y|X=1)\P(X=1) - \E(Y|X=0)\P(X=0) - \P(X=1)
\]
and
\[
    \E(C_0)\leq \E(Y|X=1)\P(X=1) + \P(X=0) - \E(Y|X=0)\P(X=0).
\]
The conclusion is that $\E(C_1)-\E(C_0)$ lies between
\[
    \E(Y|X=1)\P(X=1)-\E(Y|X=0)\P(X=0) - \P(X=1)
\]
and
\[
    \E(Y|X=1)\P(X=1)-\E(Y|X=0)\P(X=0) +\P(X=0)
\]
and this interval has diameter $1$.

\end{enumerate}
\end{document}
